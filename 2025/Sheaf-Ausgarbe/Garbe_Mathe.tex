\Large\textbf{%Artikeltitel
Was ist eine Garbe?%
}

\medskip\normalsize%Artikeltext
\textbf{Definition (Garbe):}
Sei $X$ ein topologischer Raum
Eine \emph{Garbe} $\mathcal{F}$ \emph{abelscher Gruppen} auf $X$ besteht aus folgenden Daten:
\vspace{-\topsep}
\begin{enumerate}
	\item Für jede offene Menge $U$ eine abelsche Gruppe $\mathcal{F}(U)$.
	\vspace{-\topsep}
	\item Für jede Inklusion offener Mengen $V\subseteq U$ ein Gruppenhomomorphismus
	$\mathrm{res}^U_V:\mathcal{F}(U)\rightarrow\mathcal{F}(U)$.
\end{enumerate}
\vspace{-\topsep}
Wir fordern, dass die Axiome (Funktiorialität), (Lokalität) und (Kleben) gelten; siehe Wikipedia oder ncatlab.
\bigskip 

\textbf{Beispiele und Nicht-Beispiele:} 
Sei $X:=\mathbb{R}$ und $\mathcal{F}$ gegeben durch $\mathcal{F}(U):=\{f:U\rightarrow\mathbb{R}\mid f\text{ \say{blabla}}\}$ und $\mathrm{res}^U_V(f)=f|_V$.\\
Für \say{blabla} $\in$ \{konstant, beschränkt\} gilt (Kleben) nicht.\\
Für \say{blabla} $\in$ \{lokal konstant, lokal beschränkt, stetig, analytisch\} ist dies eine Garbe. 
\bigskip 

\textbf{Lemma (Eine Garbe besteht aus Halmen): }
Sei $\mathcal{F}$ eine Garbe abelscher Gruppen auf $X$.
Dann ist die kanonische Funktion\footnote{siehe Literatur} $\mathcal{F}(U)\rightarrow\prod_{x\in U} \mathcal{F}_x$ injektiv (die Gruppen $\mathcal{F}_x$ nennt man Halme).

%nach kanonische Funktion:"\footnote{siehe Literatur}"

\begin{flushright}%Autor
Alex
\end{flushright}
