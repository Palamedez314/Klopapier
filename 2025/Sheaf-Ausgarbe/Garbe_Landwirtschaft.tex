\Large\textbf{%Artikeltitel
Was ist eine Garbe?%
}

\medskip\normalsize%Artikeltext
Ein \emph{Garbe} (auch \emph{Bürde}) ist ein Bündel aus Getreidehalmen, dass zur Trocknung zusammengebunden wurde.
Dabei wurden oft mehrere Garben zusammen in \emph{Diemen} (auch \emph{Docke, Puppe, Stiege, Feime, Triste, Heinze, Hock} oder \emph{Höcke} genannt) aufgestellt und nach dem Trocknen gedroschen.
Ursprünglich wurden Garben von Hand mit Halmen oder Garn gebunden, später erleichterte dies jedoch die Erfindung des Mähbinders.
Heutzutage findet man nur noch wenig Garben in der Landwirtschaft: Seit ca. den 1960er Jahren wartet man bis zur Totreife des Getreides, das dann direkt von Mähdreschern gedroschen wird.

\begin{flushright}%Autor
Benedikt
\end{flushright}
