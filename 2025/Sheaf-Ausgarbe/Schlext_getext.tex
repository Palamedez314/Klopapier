\Large\textbf{%Artikeltitel
	SCHL\kern-.1em%
	\lower0.5ex\hbox{E}\kern-.12em%
	\!%
	\rotatebox[origin=c]{45}{\huge$\mathbf{+}$}\!\!%
	T GET\kern-.1em%
	\lower0.5ex\hbox{E}\kern-.12em%
	\!%
	\rotatebox[origin=c]{45}{\huge$\mathbf{+}$}\!\!%
	T%
}

\medskip\normalsize%Artikeltext
Oft wird \verb$\times$ durch ein \verb$x$ ersetzt.
Mit \verb$\rotatebox$ braucht es auch kein Plus mehr! Hier i\!\!\! \rotatebox[origin=c]{90}{$<$} schiefe Variationen von Plus!
\begin{align*}
42 \mathrel{\rotatebox[origin=c]{50}{\textsf{x}}} 17 \mathrel{\rotatebox[origin=c]{130}{\textsf{x}}} 7 \mathrel{\rotatebox[origin=c]{230}{\textsf{x}}} 3 \mathrel{\rotatebox[origin=c]{310}{\textsf{x}}} \mathrm{e}^{\mathrm{i}\pi} = 69 \mathrel{\rotatebox[origin=c]{90}{\textsf{\i}}} 1 
\end{align*}
Quizfrage: Wie ist das Minus entstanden?
\begin{flushright}%Autor
	\rotatebox[origin=c]{180}{W}oritz
\end{flushright}


