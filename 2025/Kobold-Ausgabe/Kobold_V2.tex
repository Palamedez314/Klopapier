\Large\textbf{%Artikeltitel
%Die Leiden der kleinen Jasmin:
Mein Leben mit 157 cm%
}

\medskip\normalsize%Artikeltext
Die durchschnittliche Körpergröße eines Mannes in Deutschland beläuft sich auf 178,9 cm. Die durchschnittliche deutsche Frau ist 165,8 cm groß.  Mit meinen 157 cm liege ich also deutlich unter dem Schnitt. Da fragt manch einer mich (mit gesenktem Blick natürlich): \glqq Wie lebt es sich so als kleiner Mensch?\grqq\\
%Um eine angemessene Antwort auf diese Frage zu finden, achte ich seit ein paar Wochen
Seit ein paar Wochen achte ich im Alltag mehr auf die Auswirkungen meiner Größe. Seitdem fallen mir am laufenden Fließband Nachteile auf. Ich komme zuhause ohne Stuhl nicht an die Gewürze. Beim Vorrechnen in der Übungsgruppe beschreibe ich nur drei Viertel der Tafel.
%, da ich das obere Viertel nicht erreiche.
In großen Menschengruppen gehe ich leicht unter und man verliert mich aus den Augen.
Allerdings haben sich auch ein paar Vorteile bemerkbar gemacht. Beim Lauftreff ziehe ich am steilen Anstieg easy an den großen Männern vorbei. Im Zug oder Auto macht mir mangelnde Beinfreiheit überhaupt nichts aus. Zudem zahle ich manchmal nur den Kinderpreis.
Es ist also nur eine Frage der Perspektive, ob die Vorzüge oder Nachteile überwiegen. 
%Ich habe eine aus meiner Sicht passende Antwort auf die eingangs gestellte Frage gefunden: 
Meine Antwort auf die eingangs gestellte Frage:
%gefunden:
\glqq Ich bin nicht klein, ich bin nur auf das Beste reduziert\grqq.

\begin{flushright}%Autor
Jasmin
\end{flushright}
