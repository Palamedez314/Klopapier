\Large\textbf{%Artikeltitel
Dieser Text ist dreist%
}

\medskip\normalsize%Artikeltext
In der Bold-Ausgabe des Klopapiers darf natürlich kein Bold-Text fehlen. Aber was bedeutet das Wort bold überhaupt? Klein-Jasmin lernte dieses Adjektiv erstmals in der 6. Klasse im Zusammenhang eines Singspiels kennen. Damals sangen wir von einem Esel: \glqq He is brave and he is bold\grqq . Ich fragte mich, was das Wort bedeutete. Mein Lehrer antwortete, es heiße mutig. Für die naive kleine Jasmin ein positiv konnotiertes Adjektiv. Mutig, aus sich herauskommen, seine Ängste überwinden.

\bigskip An dieser positiven Assoziation hielt ich jahrelang fest, bis zur \Klopapier-Sitzung der Bold-Ausgabe. In dieser Sitzung befragten wir leo.org nach der Übersetzung von bold. Der Übersetzer spuckte Wörter wie kühn, frech und dreist aus. Nur negative Assoziationen. Das heißt meine Aufgabe, ist es leider nicht, einen aufbauenden und ermutigenden Text zu schreiben. Nein, der Text soll dreist und frech sein. Was ist wohl das Dreisteste, was man in einer Bold-Ausgabe behaupten könnte? Ich weiß es: Dicke Kreide ist besser als dünne.

\begin{flushright}%Autor
Jasmin
\end{flushright}
