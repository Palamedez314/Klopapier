\Large\textbf{%Artikeltitel
	SCHL\kern-.1em%
	\lower0.5ex\hbox{E}\kern-.12em%
	XT GET\kern-.1em%
	\lower0.5ex\hbox{E}\kern-.12em%
	XT%
}

\medskip\normalsize%Artikeltext
Wenn man in Latex fett schreiben will, ergeben sich mehrere Möglichkeiten.
Man kann natürlich normal einfach \textbf{\\textbf\{Text\}} o.ä. benutzen. Man kann aber auch lustiger sein:\\
\begin{itemize}
	\setlength\itemsep{1em}
	\item \makebox[0.4pt][l]{\textbackslash makebox[0.4pt][l]\{Text\}Text}\textbackslash makebox[0.4pt][l]\{Text\}Text\\
	\item \tikz{\node at (0,0) {\\tikz\{\\node at (0,0) \{Text\}; \\node at (0.015,0) \{Text\}\}}; \node at (0.015,0) {\\tikz\{\\node at (0,0) \{Text\}; \\node at (0.015,0) \{Text\}\}}}
	\item \fontfamily{LibreBodoni-TLF}\selectfont
	\textbackslash fontfamily\{LibreBodoni-TLF\} \textbackslash selectfont 
	
	Text
	
	\textbackslash fontfamily\{cmr\} \textbackslash selectfont
	\fontfamily{cmr}\selectfont
\end{itemize}


Dies geht natürlich auch mit anderen Fonts, die fettgedruckt aussehen.

\begin{flushright}%Autor
	Benedikt \& Alex
\end{flushright}


