\Large\textbf{%Artikeltitel
	FÖJ: Ein Selbstinterview%
}

\medskip\normalsize%Artikeltext

\textit{Bernhard, du machst nun also ein FÖJ. Was ist das überhaupt?}

Ein FÖJ ist ein freiwilliges ökologisches Jahr. So ähnlich wie ein FSJ, also freiwilliges soziales Jahr, nur eben mit ökologischem Fokus. Außerdem gibt es den Bundesfreiwilligendienst, BFD. Dieser ist etwas freier und ohne Altersbeschränkung. Personen im FÖJ oder FSJ müssen das gesamte Jahr echt jünger als 27 sein.\\

\vspace*{-1mm}
\textit{Und was passiert da so?}

Das unterscheidet sich von Einsatzstelle zu Einsatzstelle. Ich mache Jugendbildungsarbeit, betreue Gruppen von Ehrenamtlichen und mache Hintergrundarbeit am Schreibtisch. Andere machen Workshops mit Schulklassen, helfen bei der Landwirtschaft oder beobachten Vögel. Außerdem gibt es die Möglichkeit, ein eigenes Projekt auszudenken und umzusetzen.\\

\vspace*{-1mm}
\textit{Wem würdest du ein FÖJ empfehlen?}

Allen, die mal etwas anderes machen wollen. Oder die nicht wissen, was sie gerade machen sollen. Ein FÖJ ist eine gute Möglichkeit, sich auszuprobieren und dabei etwas Sinnstiftendes zu machen. Es ist eine Zeit, die zum Reflektieren und Orientieren genutzt werden kann. Es gibt regelmäßig Seminarwochen, die dazu einen Rahmen bieten, und die Möglichkeit, Praktika zu machen oder Infotage zu besuchen.\\

\vspace*{-1mm}
\textit{Wie ist das mit dem Geld?}

Es gibt ein geringes Taschengeld. Manche Einsatzstellen bieten Unterkunft oder Verpflegung. Je nach Situation gibt es die Möglichkeit, Wohngeld oder Bürgergeld zu beantragen.\\

\vspace*{-1mm}
\textit{FÖJ nach dem Studium, ist das nicht ungewöhnlich?}

Tatsächlich machen die meisten ein FÖJ nach der Schule. Ich bin auch mit Abstand die älteste Person in meiner Seminargruppe. Ich denke schon, dass einige Erfahrungen für mich nicht so neu sind wie für andere. Gleichzeitig ist es auch schön, aus Studium und Engagement einiges an Erfahrung ins FÖJ mitzubringen.\\

\vspace*{-1mm}
\textit{Wie verwendest du Mathematik im FÖJ?}

Auf den ersten Blick kaum. Ich baue manchmal Knoten aus Fidget-Toys – \textit{lacht}. Und gleichzeitig mag ich es weiterhin, Probleme zu lösen und Zusammenhänge zu verstehen.

%% unten Auskommentieren?
\begin{flushright}%Autor
	Bernhard
\end{flushright}
