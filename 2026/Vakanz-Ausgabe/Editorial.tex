%Diese Datei darf nicht umbenannt werden!
Liebe Mitmenschen!

\bigskip
\vspace*{-1mm}
Die leere Menge ist im Grunde unglaublich einfach:\\
Wir nennen eine Menge leer, wenn sie keine Elemente hat ($\forall x\in M: \text{Falsch}$) und nach Extensionalität von Mengen gibt es auch nur eine solche Menge $\varnothing$.\\
%Aus der leeren Menge gibt es in jede andere Menge $B$ eine Abbildung, weil wir dafür nur für alle $x\in \varnothing$ ein $y\in B$ angeben müssen... wir müssen also gar nichts tun!\\
%Nach Extensionalität von Abbildungen ist diese Abbildung $\varnothing \rightarrow B$ eindeutig. (Also $\varnothing$ initial in $Set$ :))\\
Aber was gibt es für Abbildungen in die leere Menge? Für eine solche Abb. $f: A \rightarrow \varnothing$ gilt: $\exists x \in A \Rightarrow \exists x \in \varnothing \Rightarrow \text{Falsch}$. Also ist (das eindeutige!) $\varnothing \rightarrow \varnothing$ die einzige solche Abbildung.\\
Wir bekommen also $\exists x \in X \Leftrightarrow \neg \exists x \in \varnothing^{X} := \{X \rightarrow \varnothing\}$\\

\vspace*{-1mm}
Und hier fängt der Spaß an: Eine logische Aussage könnten wir  durch eine Menge neu definieren, so dass unsere Aussage dann wahr ist, wenn die Menge ein Element hat (z.B.: Wahr := $X$ mit $\exists x \in X$, Falsch := $\varnothing$) und unsere alt bekannten Logischen Verknüpfungen finden wir dann durch Überlegungen wie eben: $\neg P := \text{Falsch}^{P},\quad P \land Q := P \times Q \quad P \lor Q := P \amalg Q \quad P \Rightarrow Q := \{P \rightarrow Q\}, \quad \forall x \in X: P(x) := \prod_{x\in X}P(x), \quad$ usw...\\

\vspace*{-1mm}
Aber warum das ganze? Weil wir jetzt plötzlich unser Logiksystem verändern können indem wir unser Mengen-Modell verändern:\\
Was passiert wenn wir bei unseren Überlegungen nur einelementige Mengen nehmen (und dafür z.B.: $\lor$ umdefinieren)?\\
Was wenn unsere Mengen topologische Räume, und die Abbildungen stetig sind? Plötzlich betreiben wir konstuktive Mathematik!\\

\vspace*{-1mm}
Alles nur weil wir die leere Menge mal genauer angeschaut haben...

\vspace*{-1mm}
\bigskip Ich wünsche weiter viel Spaß \& Erfolg beim Lesen \& Kacken.
\vspace*{-1mm}
\begin{flushright}
Jonny
\end{flushright}