\documentclass[twoside,11pt,a4paper]{article} % weitere Option: twoside

% Aus der Klopapier-Vorlage, zuletzt angepasst am 23.11.2023

%------------------------------------------------------------------------------------------------
%
%  ES FOLGEN VOREINSTELLUNGEN, DIE IN DER REGEL NICHT GEÄNDERT WERDEN MÜSSEN
%
%------------------------------------------------------------------------------------------------


% weitere Pakete laden
\usepackage[utf8]{inputenc}		% Kodierung einstellen
\usepackage[ngerman]{babel}
\usepackage[T1]{fontenc}
\usepackage{anyfontsize}

\usepackage{courier}            % Courier-Font (für \texttt{ })

\usepackage[normalem]{ulem}     % Unter-/Durchstreichungen

\usepackage[gen]{eurosym}       % Euro-Symbol "€", \euro{}

\usepackage{amsmath}            % AMS (American Mathematical Society): diverse
\usepackage{amssymb}            % Weitere AMS Symbole
\usepackage{amsfonts}           % Weitere AMS Zeichensätze

\usepackage{color}              % Für Farben. Macht \kloblau usw. möglich.
\usepackage{graphicx}           % Für Bilder.
\usepackage{float}              % Irgendwas mit Position von Table/Figure/...
\usepackage{geometry}           % Um Raender anzupassen.

\usepackage{lipsum}				% Für Blindtext.

\usepackage{lettrine}           % Für einen grossen Anfangsbuchstaben

\usepackage[hidelinks]{hyperref}% Links und so

% Eigene Farben
\definecolor{kloblau}{rgb}{0,0.3,0.5}
%\definecolor{uniblau}{cmyk}{1,0.55,0,0.43}
\definecolor{klorange}{rgb}{0.8,0.5,0} %Ganz recht! Klorange! Mit nur einem o. 
\definecolor{klogruen}{rgb}{0,0.5,0.3}
\definecolor{klorot}{rgb}{0.5,0,0.3}
\definecolor{kloviolett}{rgb}{0.3,0,0.5}

%Keine Leerstelle nach Kommas
\DeclareMathSymbol{,}{\mathord}{letters}{"3B}

% Passt die Textraender an
\newgeometry{
  left= 1cm,
  right=1cm,
  top=1cm,
  bottom=1cm,
  bindingoffset=0mm
}

% Fussnotennummerierung in Minipage
\renewcommand{\thempfootnote}{\arabic{mpfootnote}}

\setlength{\parindent}{0pt}

% Klopapier-Logo
\newcommand{\Klogo}{\mathfrak{Klo}\mathtt{pap\mspace{-0.5mu}i\mspace{-1mu}er}}
% Klopapier in texttt
\newcommand{\Klopapier}{\texttt{Klopapier}}


%------------------------------------------------------------------------------------------------
%
%  AB HIER SOLLEN AENDERUNGEN VORGENOMMEN WERDEN
%
%------------------------------------------------------------------------------------------------

%  AUSGABENSPEZIFISCHE VOREINSTELLUNGEN

%Titelfarbe. 
\newcommand{\Titelfarbe}{% Optionen: kloblau, klorange, klogruen, klorot, kloviolett
kloviolett%
}


\newcommand{\TitelfarbeZwei}{% Optionen: kloblau, klorange, klogruen, klorot, kloviolett
klogruen%
}
%Ausgaben-Titel
\newcommand{\Titel}{% Hier Titel einfügen
Vakanz-Ausgabe
}

%Titelbild
\newcommand{\Titelbild}{% Hier Dateiname einfügen.
Moritz_lila_cropped.jpeg
}

\newcommand{\TitelbildZwei}{% Hier Dateiname einfügen.
Moritz_grün_cropped.jpeg
}

%Titelbildbeschreibung
\newcommand{\Titelcaption}{% Hier Titelbild beschreiben.
Illustration: Tabea
}

%----------------------------------------
%  ARTIKEL
%----------------------------------------

%----------------------------------------
%  ARTIKEL LINKS
%----------------------------------------
%Erster Artikel links
%
%Artikeltext
\newcommand{\linksArtikelEins}{%
bunte_Eminenz.tex%
}
%Artikelfarbe
\newcommand{\linksArtikelEinsFarbe}{%
klogruen%
}

%Zweiter Artikel links
%
%Artikeltext
\newcommand{\linksArtikelZwei}{%
%
}
%Artikelfarbe
\newcommand{\linksArtikelZweiFarbe}{%
klorot%
}

%Dritter Artikel links
%
%Artikeltext
\newcommand{\linksArtikelDrei}{%
%
FÖJ.tex}
%Artikelfarbe
\newcommand{\linksArtikelDreiFarbe}{%
kloblau%
}

%----------------------------------------
%  ARTIKEL RECHTS
%----------------------------------------
%Erster Artikel rechts
%
%Artikeltext
\newcommand{\rechtsArtikelEins}{%
schlext_getext.tex%
}
%Artikelfarbe
\newcommand{\rechtsArtikelEinsFarbe}{%
klorange%
}

%Zweiter Artikel rechts
%
%Artikeltext
\newcommand{\rechtsArtikelZwei}{%
Rätsel.tex%
}
%Artikelfarbe
\newcommand{\rechtsArtikelZweiFarbe}{%
kloblau%
}

%Dritter Artikel rechts
%
%Artikeltext
\newcommand{\rechtsArtikelDrei}{%
ArtikelVolltext2%
}
%Artikelfarbe
\newcommand{\rechtsArtikelDreiFarbe}{%
kloviolett%
}

%Vierter Artikel rechts
%
%Artikeltext
\newcommand{\rechtsArtikelVier}{%
ArtikelMitBildStattArtikel%
}
%Artikelfarbe
\newcommand{\rechtsArtikelVierFarbe}{%
kloviolett%
}

%Fünfter Artikel rechts
%
%Artikeltext
\newcommand{\rechtsArtikelFuenf}{%
ArtikelVolltext3%
}
%Artikelfarbe
\newcommand{\rechtsArtikelFuenfFarbe}{%
klorange%
}

%------------------------------------------------------------------------------------------------
%
%  AB HIER MUESSEN EIGENTLICH KEINE AENDERUNGEN MEHR VORGENOMMEN WERDEN
%
%------------------------------------------------------------------------------------------------

\begin{document}
\pagestyle{empty}
 %%%%%%%%%%%%%%%%%%%%%%%%%%%%%%%%%%%%%%%%%%%%%%%%%%%%%%%%%%%%%%%%%%%%%%%%%%%%%%%%%%
 % Hier linke Seite
 %%%%%%%%%%%%%%%%%%%%%%%%%%%%%%%%%%%%%%%%%%%%%%%%%%%%%%%%%%%%%%%%%%%%%%%%%%%%%%%%%%
\begin{center}
\begin{minipage}{0.6\textwidth}
\begin{center}
 \fontsize{70}{70}\selectfont
 \textcolor{\Titelfarbe}{$\Klogo$}
 \normalsize
\end{center}
%Diese Datei darf nicht umbenannt werden!
Liebe Mitmenschen!

\bigskip Aus gegebenem Anlass hab ich 

\bigskip Ich wünsche viel Spaß \& Erfolg beim Lesen \& Kacken.
\begin{flushright}
Moritz
\end{flushright}
\vspace*{-1mm}
\color{\linksArtikelEinsFarbe}
\input{\linksArtikelEins}
\normalcolor
\vspace*{-1mm}

\color{\linksArtikelZweiFarbe}
\input{\linksArtikelZwei}
\normalcolor

%\color{\linksArtikelDreiFarbe}
%\input{\linksArtikelDrei}
%\normalcolor
\end{minipage}
 \hspace{0.05\textwidth}
 %%%%%%%%%%%%%%%%%%%%%%%%%%%%%%%%%%%%%%%%%%%%%%%%%%%%%%%%%%%%%%%%%%%%%%%%%%%%%%%%%%
% Hier rechte Seite
 \begin{minipage}{0.3\textwidth}
% Hier Ausgabenbild mit Benennung
\begin{minipage}{0.1\textwidth}
 \rotatebox{90}{\kern 0cm \textcolor{\Titelfarbe}{\huge{\texttt{\Titel}}}}
\end{minipage}
\begin{minipage}{0.85\textwidth}
 \begin{figure}[H]
\includegraphics[width=\textwidth]{\Titelbild}
\end{figure}
\tiny \hfill \Titelcaption
\end{minipage} 

\vspace{0.5cm}

\begin{tabular}{|p{0.15\textwidth}p{0.7\textwidth}|}
\hline & \\[-1.5ex]
\multicolumn{2}{|l|}{\textbf{Termine:}} \\
%Diese Datei darf nicht umbenannt werden!
%
%Hier die Termine der Form
%Datum & Termin \\
%eintragen. Der letzte Termin erhält kein \\
01.01. & Ein Termin \\
01.02. & Ein anderer Termin \\
02.03. & Noch ein Termin \\
03.05. & Termine ohne Ende\\
05.08. & Terminologie\\
08.13. & Termine der Zukunft!\\
13.21. & Terminierung!
%Der letzte Termin erhält kein \\\\[1ex]
\hline
\end{tabular}

\vspace{0.5cm}

\color{\rechtsArtikelEinsFarbe}
\input{\rechtsArtikelEins}
\normalcolor

\color{\rechtsArtikelZweiFarbe}
\input{\rechtsArtikelZwei}
\normalcolor
\end{minipage}

\vspace{0.5cm}

\Large{\textcolor{\Titelfarbe}{Lob \& Kritik bitte an \textbf{klopapier@mathe.stuvus.uni-stuttgart.de}.}}\\
\tiny{Die Texte stellen jeweils die Meinung der Autor*innen und nicht notwendigerweise die der Fachgruppe dar.}
\end{center}
 
 %% Zweite Seite %% 
 
\newpage

\begin{center}
\begin{minipage}{0.3\textwidth}
% Hier Ausgabenbild mit Benennung
\begin{minipage}{0.85\textwidth}
\begin{figure}[H]
\includegraphics[width=\textwidth]{\TitelbildZwei}
\end{figure}
\tiny \hfill \Titelcaption
\end{minipage} 
\begin{minipage}{0.1\textwidth}
\rotatebox{-90}{\kern 0cm \textcolor{\TitelfarbeZwei}{\huge{\texttt{\Titel}}}}
\end{minipage}

\vspace{0.5cm}

%\begin{tabular}{|p{0.15\textwidth}p{0.7\textwidth}|}
%\hline & \\[-1.5ex]
%\multicolumn{2}{|l|}{\textbf{Termine:}} \\
%%Diese Datei darf nicht umbenannt werden!
%
%Hier die Termine der Form
%Datum & Termin \\
%eintragen. Der letzte Termin erhält kein \\
01.01. & Ein Termin \\
01.02. & Ein anderer Termin \\
02.03. & Noch ein Termin \\
03.05. & Termine ohne Ende\\
05.08. & Terminologie\\
08.13. & Termine der Zukunft!\\
13.21. & Terminierung!
%Der letzte Termin erhält kein \\\\[1ex]
%\hline
%\end{tabular}

\vspace{0.5cm}

\color{\rechtsArtikelDreiFarbe}
\input{\rechtsArtikelDrei}
\normalcolor

\color{\rechtsArtikelVierFarbe}
\input{\rechtsArtikelVier}
\normalcolor

\color{\rechtsArtikelFuenfFarbe}
\input{\rechtsArtikelFuenf}
\normalcolor

\end{minipage}
\hspace{0.05\textwidth}
\begin{minipage}{0.6\textwidth}
\begin{center}
	\fontsize{70}{70}\selectfont
	\textcolor{\TitelfarbeZwei}{$\Klogo$}
	\normalsize
\end{center}
%%Diese Datei darf nicht umbenannt werden!
Liebe Mitmenschen!

\bigskip Aus gegebenem Anlass hab ich 

\bigskip Ich wünsche viel Spaß \& Erfolg beim Lesen \& Kacken.
\begin{flushright}
Moritz
\end{flushright}
%\color{\linksArtikelEinsFarbe}
%\input{\linksArtikelEins}
%\normalcolor
%
%\color{\linksArtikelZweiFarbe}
%\input{\linksArtikelZwei}
%\normalcolor

\color{\linksArtikelDreiFarbe}
\input{\linksArtikelDrei}
\normalcolor
\end{minipage}



\vspace{0.5cm}

\Large{\textcolor{\TitelfarbeZwei}{Lob \& Kritik bitte an \textbf{klopapier@mathe.stuvus.uni-stuttgart.de}.}}\\
\tiny{Die Texte stellen jeweils die Meinung der Autor*innen und nicht notwendigerweise die der Fachgruppe dar.}
\end{center}

\end{document}
